\documentclass[12pt]{article}
\usepackage[utf8]{inputenc}
\usepackage{amssymb}
\usepackage{graphicx,subfigure}              % with figures
% sometimes needed to have pdf files 
\pdfsuppresswarningpagegroup=1
\topmargin -1mm
\oddsidemargin -1mm
\evensidemargin -1mm
\textwidth 170mm
\textheight 220mm
\parskip 2mm
\parindent 3mm
\pagestyle{empty}
\usepackage{xcolor}
\usepackage[normalem]{ulem}

% For appendices
\usepackage[titletoc,title,header]{appendix}

%\usepackage[firstpage]{draftwatermark}
%\SetWatermarkScale{4}

\begin{document}

\begin{center}

  {\Large \bf XMLTDB data format documentation}

  \bigskip
  Taichi Abe, Qing Chen, Shuanglin Chen, Nathalie Dupin, Bengt
  Hallstedt, Aurelie Jacob, Ursula R. Kattner, Lina Kjellqvist, Fabio
  Miani, Richard Otis, Alexander Pisch, Erwin Povoden-Karadeniz, Naraghi
  Reza, Malin Selleby, Bo Sundman, Axel van de Walle

  DRAFT \today

\end{center}

{\color{magenta} Use dolor to indicate changes!}

\sout{Use sout to overstrike text}

\abstract{This is the draft definition of an XML format for
  thermmodynamic data for Calphad applications in materials science}

\section{Introduction}

During the work with the 1991 unary database~\cite{91Din} a database
format called TDB was developed to handle the complex set of model
parameters used in models for different phases.  Today, almost 40
years later, it is time to take advantage of the development of
computer science to revise this format using the XML, a well
established markup language, explained for example in the web
site~\cite{XML}.

During its 40 years of use the TDB databases have been extended to
include other data, for example kinetic and structural, which is
related to the phases in a system.  The intention is that the XMLTDB
should also accomodate such data.  It should be simple to introduce
additional model parameters related to any other property depending on
the phase constitution, $T$ and $P$.

\bigskip
The basic structure in the TDB format is:

{\bf keyword data !}\\

\noindent
where the exclamation mark, ``!'' is the terminator, This makes it
possible to transform to XML by using a mix of XML elements and
attributes for the data.  The transfer to XML has been straightforward
but not painless as a complex set of additional features has been
introduced during the time TDB files has been used.  Most of the data
in a TDB file can be tranferred to a tentative XMLTDB format using an
UPLOAD program~\cite{upload}.  An example is found in
Appendix~\ref{sc:alfe}.

The XMLTDB format should contain the thermodynamic and related data
but software specific directives, intended to simplify calculations
using the data, should be excluded or included in a software specific
XML element.

\subsection{The Gibbs energy expression}

The Calphad method to calculate phase diagrams based on modeling the
Gibbs energy of each phase in a system has been described in many
books and papers~\cite{70Kau,98Sau,08Hil,09Luk, 14Cam,
  14Kau,19Pel,21Sun}.  The thermodynamic database provides model
parameters to a thermodynamic software in order to calculate the Gibbs
energy, $G^{\alpha}_M$ for each phase in a system.  The main objective
is to calculate equilibrium state of the system by minimizing the
total Gibbs energy of the system, depending on the set of external
condition and the model parameters has been fitted to experimental and
theoritcal data in a separate assessment procedure.  A more
practically useful application is to provide thermodynamic information
for simulating phase transformations together with kinetic data.

For each phase the Gibbs energy is expressed as:
\begin{eqnarray}
  G^{\alpha}_M &=& ~^{\rm srf}G^{\alpha}_M - T~^{\rm cfg}S^{\alpha}_M + ~^EG^{\alpha}_M + f^{\alpha}_M \cdot ~^{\rm phy}G^{\alpha}_m  \label{eq:generic}
\end{eqnarray}
where the term $~^{\rm srf}G^{\alpha}_M$ has parameters for the
surface of reference of the Gibbs energy and $~^EG^{\alpha}_M$ has
parameters to describe interaction energies between constituents
within the phase.  The capital $M$ used as subscript means the Gibbs
energy is per mole formula unit of the phase, which may vary with the
phase constitution.  Each phase can have a different model, in
particular for its configurational entropy, $~^{\rm cfg}S_M$, which
depend on its structure.  There can be conytribution from several
physical models, denoted $~^{\rm phy}G_m$, with specific model
parameters. This term has subscript, ``$m$'', as this is normally
modeled per mole of atoms and the factor $f^{\alpha}_M$ is the number
if atoms per formula unit of the phase.

\subsection{Assessment of model parameters}

The model parameters in the database are obtaind by an assessment
procedure where experimental and theoretical data are recalculated
from the models using a thermodynamic software.  The model parameters
are varied using a least square fitting software in order to obtain
the best description.  Frequently conflicting information must be
weighted or rejected by the scientist performing the assessment in
order to obtain a satisfactory result, as discussed in many published
papers and also in~\cite{09Luk}.

The assessments usually deal with binary and ternary systems where
there are sufficient reliable experimental data but the main
applications of the databases are to calculate multicomponent systems
with up to 10-15 components.  It is thus important to use models which
give realistic extrapolations and the development of new models is
still ongoing, inspired by the use of the database.

The assessment procedure is an integrated process involving
experimental work, theoretical calculations and improved modeling by
scientists.  The models cannot be too complex because for example a
simulation of a phase transformations for an alloy with 10-15
components requires a very large number of equilibrium calculations
which typically can take weeks to perform on a high speed computer.
Some model parameters must also be estimated from experience because
there are no experimental data available.  But the calculations are an
essential tool replacing many years of experimental trial and error
work in order to develop new or improved materials.  This technique
can be complemented, but not replaced, by recent methods inspired
machine learning~\cite{ML1,ML2}.

\section{Description of the proposed XMLTDB format}

The XMLTDB format is an attempt to make a smooth transition from the
TDB format to the XML in order to maintain the possibility to edit the
XMLTDB file manually by scientists without particular experience of
XML.  The possibility to edit TDB files manually has been a great
advantage for developing models and assessing experimental data using
the current TDB format.

\subsection{Short overview}

This summary has the basic set of XML elements for the XMLTDB format.
A detailed explanation is found in Appendix~\ref{sc:xmltdb}

\begin{enumerate}
\item {\bf Defaults} specify default values for some attributes in
  order to avoid listing irrelevant information.
  
\item {\bf Models} is an optional part of the XML file with
  documentation and references for various models used.  A separate
  XMLTDB model library file with all models and model parameter
  identifiers (MPID) will be provided.  Each model has an ID which can
  be used whan adding this model to a phase.  New models will be added
  when tested and available in the application software.
  
\item {\bf Element} specifies an element with its chemical symbol, the
  name of its reference phase, its molar mass, H298 and S298.
  
\item {\bf Species} is an element or a molecular like aggregate of two
  or more elements with fixed stoichiometric ratios.  A species has a
  name which is used when it is a constituent of a phase.  A species
  can have an electric charge.  Some species used for phases with the
  MQMQA~\cite{01Pel2} or UNIQUAC~\cite{20Li} models will have
  additional data such as bond fractions or volume and area needed for
  the configurational entropy of the phase.
    
\item {\bf TPfun} has an ID and a mathematical expression in $T$ and
  $P$ and other TPfuns.  Only very simple expressions are allowed as
  explained in Appendix~\ref{sc:expression}.
  
\item {\bf Trange} is needed if a TPfun or parameter is valid in
  limited $T$ range or have several ranges.  Each Trange record with
  the upper $T$ limit and expression.  The value of the expression and
  its first and second derivative wrt $T$ must be continuous when
  crossing a range.
  
\item {\bf Phase} has a name, a configurational entropy model and all
  model parameters are associated with a phase.  It can have one or
  more sublattices with different number of sites and constituents.
  Additional models can contribute to its Gibbs energy.
  
\item {\bf Sublattices} specifies the number of sublattices of a phase
  and their site ratios.  Optionally more crystallographic information
  can be provided.
  
\item {\bf Constituents} has a sublattice index and a list of species
  as constituents in the sublattice.
  
\item {\bf Amend-phase} specifies an additional model for a phase.  It
  may have some extra model specifications.
  
\item {\bf Parameter} has a model parameter identifier (MPID)
  specifying the type of property, a phase name and sublattice
  constituents.  It has also a degree to indicate some additional
  model feature and a bibliographic reference.  Finally it has a
  mathematical expression which may depend on $T$ and $P$ and include
  TPfun identifiers.

\item {\bf Bibliography} contains all the bibliographic information
  for models and parameters.
\end{enumerate}

\subsection{Manual editing and efficient parsing}\label{sc:manualedit}

There are special parsers of XML files which work better the more
detailed the data is separated in elements and attributes.  However,
very detailed specification of the data makes it inconvenient for
editing manually which is still an important tool for scientists
interested in modeling and assessment of model parameters.  For
example a model parameter can be described as:

\begin{verbatim}
<Parameter id="G(FCC,W:C,VA;0)" expression=" 50000+10*T;" bibref="90Gus"/>
\end{verbatim}
which is similar to the way the parameter is listed in the TDB format
or in an extended way which is simpler for an XML parser such as:

\begin{verbatim}
<Parameter mpid="G" phase="FCC" bibref="90Gus">
  <Constituentarray degree="0">
    <Site refid="1">
      <Constituent id="W"/>
    </Site>
    <Site refid="2">
      <Constituent id="C"/>
      <Constituent id="VA"/>
   </Site>
  </Constituentarray>
  <Trange  expression=" 50000+10*T;" />
</Parameter>
\end{verbatim}
and there is no question which is preferred by a scientist who is
editing the XML file.

However, there is no reason not to allow both methods in the XMLTDB
file, a human can without too much effort translate the extended
parameter attributes to the simpler one in order to edit it and with a
little effort the software using an XML parser can do the opposite.

\subsection{Extentions and future changes in the XMLTDB format}

The XMLTDB format is open to extensions and changes but they should be
communicated and agreed by a governing committee, for example
appointed by SGTE, in order to be universally agreed and circulated to
all groups which use XMLTDB.


\section{Application and software specifics}

The TDB files are used by many different software to calculate
eqilibria, phase diagrams and various properties of multicomponent
systems.  The databases are also used for simulations of phase
transformation together with mobilities and other phase dependent data
stored in the TDB file.

More things to write about:
\begin{enumerate}
\item How to handle duplicate parameters in the XML file?
\item A wildcard parameter is independent of the constituent in one or
  more sublattices, how to handle wildcard parameters?  In particular
  if there is an ``overlapping'' parameter with a specific constituent
  in the same sublattice as the parameter with a wildcard.
\item Order of RK constituents, alphabetical or as written?
\item Should the database specify miscibility gaps for example in
  Fe-Ti-C to have both an FCC representing austenite and an FCC
  representing the MC carbide?  Or in Al-Ni to represent the FCC and
  the L1$_2$ ordered phase.
\item There is a need for a free software to check consistency of an
  XMLTDB file.
\item Is there an interst to develop a software to write the model
  parameters in an XML file as a nicely formatted a LaTeX file?
\item For the application software: is the $f^{\alpha}_M$ from
  eq.~\ref{eq:generic} correctly implemented?
\item A property such as BMAGN is extensive but as it is inside a
  physical model I am not sure it should be multiplied with the
  $f^{\alpha}_M$ defined in eq.~\ref{eq:generic}.
\item UPLOAD program has not implemented

  - extracting software specific data from TDB.

  - DATA\_INFORMATION and some other TDB keywords.
{\{color{magenta} 
item The gas constant $R$ has changed ... should we use the new value?}
\item more?
\end{enumerate}


\section{Anything else?}

\section{Summary}

The intention is that the XMLTDB format should be used for the same
applications as the current TDB files and eventually replace them.  It
is thus important to think of all possible (and future) applications
when deciding the format.  We have also to consider the scientists who
frequently explicitly manipulate the databases in order to develop
models and software based on assessments of experimental and
theoretical data.

\begin{thebibliography}{91zzz}

\bibitem{91Din} A T D Dinsdale (1991)

\bibitem{XML} https://www.w3schools.com/xml/default.asp

\bibitem{upload} repository sundmanbo/XML at https://github.com

\bibitem{70Kau} L. Kaufman and H. Bernstein {\em Computer calculation
  of phase diagrams}, (1970) New York, Academic Press.

\bibitem{98Sau} N. Saunders and A. P. Miodownik, {\em CALPHAD
  Calculation of Phase Diagrams: A Comprehensive Guide}, Pergamon
  Materials Series. (1998) Elsevier Science Inc, New York

\bibitem{08Hil} M. Hillert, {\em Phase Equilibria, Phase Diagrams and
  Phase Transformation} 2nd ed. (2007) Cambrige Univ Press, Cambridge
  
\bibitem{09Luk} H. L. Lukas, S. G. Fries and B. Sundman {\em
  Computational Thermodynamics, the Calphad method}, (2009) Cambridge
  Univ Press, Cambridge

\bibitem{14Kau} L. Kaufman L, J. {\AA}gren, {\em CALPHAD, first and
  second generation – Birth of the materials genome}, Scr Mater (2014)
  {\bf 70} 3-–6
  
\bibitem{14Cam} C. E. Campbell, U. R. Kattner, Z-K. Liu, {\em File and
  data repositories for Next Generation Calphad}, Scripta Mat (2014)
  https://doi.org/10.1016/j.scriptamat.2013.06.013

\bibitem{19Pel} A. D. Pelton, {\em Phase Diagrams and thermodynamic
  modeling of solutions}, (2019) Elsevier Press

\bibitem{21Sun} B. Sundman, N. Dupin, B. Hallstedt, {\em Algorithms
  useful for calculating multi-component equilibria, phase diagrams
  and other kinds of diagrams} Calphad {\bf 75} (2021) 102330

\bibitem{01Pel2} A. D. Pelton, P. Chartrand and G. Eriksson, {\em The
  Modified Quasi-chemical Model: Part IV. Two-Sublattice Quadruplet
  Approximation}, Met Mat Trans A, {\bf 32A} (2001) 1409--1416

\bibitem{ML1} L. Himanen, A Geurts, A S Foster and P. Rinke {\em
  Data-Driven Materials Science: Status, Challenges and Perspectives},
  Adv Science (2019) https://doi.org/10.1002/advs.201900808

\bibitem{ML2} K. Kaufmann and K. S. Vecchio, {\em Searching for high
  entropy alloys: A machine learning approach} Acta Mat. (2020)
  10.1016/j.actamat.2020.07.065

\bibitem{20Li} J. Li, B. Sundman, J. G.M. Winkelman, A. I. Vakis,
  F. Picchioni, {\em Implementation of the UNIQUAC model in the
    OpenCalphad software}, Fluid Phase Eq {\bf 507} (2020) 112398

\bibitem{05Lu} X-G. Lu, M. Selleby and B. Sundman, {\em Implementation
  of a new model for pressure dependence of condensed phases in
  Thermo-Calc}, Calphad {\bf 29} (2005) 49--55.

\bibitem{20Sun} B. Sundman, U. R. Kattner, M. Hillert, M.  Selleby,
  J. Ågren, S. Bigdeli, Q. Chen, A. Dinsdale, B. Hallstedt, A. Khvan,
  H. Mao, R. Otis {\em A method for handling the extrapolation of
    solid crystalline phases to temperatures far above their melting
    point} Calphad {\bf 68} (2020) 101737

\bibitem{01Pel} A. D. Pelton, {\em A General ``Geometric''
  Thermodynamic Model for Multicomponent Solutions}, Calphad {\bf 25}
  (2001) 319--328

\bibitem{CEF} M. Hillert, {\em The compound energy formalism},
  J All. and Comp. {\bf 320} (2001) 161--176
  
\bibitem{60Koh} F. Kohler, Monatsh Chem, Vol 91 (1960) 738--740
  
\bibitem{65Toop}  G. W. Toop, Trans Metall Soc, AIME  Vol 233 (1965) 850--854
  
\bibitem{01Chen} Q. Chen, B. Sundman, {\em Modeling of thermodynamic
  properties for Bcc, Fcc, liquid, and amorphous iron}, J. Phase
  Equilibria. {\bf 22} (2001) 631–-644.

\bibitem{12Xiong} W. Xiong, Q Chen, P. K. Korzhavyi and M. Selleby,
  {\em An improved magnetic model for thermodynamic modeling},
  Calphad, {\bf 39} (2012) 11--20.

\bibitem{14Becker} C. A. Becker, J. Ågren, M. Baricco, Q. Chen,
  S. A. Decterov, U. R. Kattner, J. H. Perepezko, G. R. Pottlacher,
  and M. Selleby, {\em Thermodynamic modelling of liquids : CALPHAD
    approaches and contributions from statistical physics} Physica
  status solidi. B, Basic research {\bf 251(1)} (2014) 33--52

\bibitem{Kopp-Neumann} H. Kopp and T. Graham, {\em II. Investigations
  of the specific heat of solid and liquid bodies},
  Proc. R. Soc. Lond. (1864) 13229–-13239
  http://doi.org/10.1098/rspl.1863.0054


\end{thebibliography}

\newpage

\begin{appendices}
\setcounter{equation}{0}
\renewcommand{\theequation}{A\arabic{equation}}
\setcounter{figure}{0}
\renewcommand{\thefigure}{A\arabic{figure}}
\section{The XML elements and attributes for Calphad databases}\label{sc:xmltdb}

This is a complete description of the XML format which will
(hopefully) replace the current TDB format.

\subsection{Some general features}

In most cases the thermodynamic data in the XMLTDB files will be the
same as in the TDB files.  A difference with the TDB file is that the
XML elements and attributes are case sensitive and cannot be
abbreviated.  The data in the TDB files is case insensitive and
keywords and phase names can be abbreviated.


\begin{enumerate}
\item All names, identifiers, symbols etc. of elements, species,
  phases, TPfuns in the XMLTDB file are case insentive.  All names
  etc.  must start with a letter A-Z and may contain letters, numbers
  0-9 and the underscore chatacter ``\_''.  Some names, as specified
  below, can contain additional characters.  In most cases this is the
  same way names are treated in TDB files.

  \begin{enumerate}
  \item An element is one or two letters which are case insensitive.
    Fictitous elements can be used.
  \item A species name can in addition contain the characters ``+'',
    ``-'' and ``/''.
  \item A model parameter identifier can in the TDB file contain an
    ampersand ``\&'' to specify a constituent.  This is used for
    mobilites at present but may be extended.  As ``\&'' is reserved
    in XML files another letter should be used, maybe``@''?
  \item ?? any more special characters or exceptions?
  \end{enumerate}

\item Names of elements and species must not be abbreviated but frequently
  phase names are and it is allowed as long as they do not become
  ambiguous.  

  Species names must not be abbreviated in the XMLTDB file and thus
  one may have species names which are abbreviations of other species,
  for example C1O and C1O2.  A species name is independent of its
  stoichiometric formula.

  Phase names can be abbreviated and each part of a phase name
  separated by an underscore, ``\_'', can be abbreviated separately.
  Interactive entering of a phase name should allow the hyphen, ``-'',
  to be used instead of the underscore.  For his reason it is not
  allowed to have phase names that are abbreviations of another phase,
  for example U3O8\_S and U3O8\_S2.

  Model parameter identifiers (MPID), TPfun names and bibrefs must not
  be abbreviated.

  A maximum length of names other identifiers should be decided, I
  propose:
  \begin{enumerate}
  \item Elements 2 letters.
  \item Element reference phase: 24 characters
  \item Species names: 24 characters.
  \item Phase names: 24 characters
  \item TPfun names: 16 characters
  \item Model parameter identifiers: 8 characters (not including
    any constituent specifier used for mobilities etc.)
  \item Model identifier: 16 characters
  \item Bibrefs: 16 characters
\end{enumerate}

\item Model parameter identifiers (MPID) are used for model parameters
  of a phase which depend on the constitution of the phase and
  frequently also on $T$ and $P$.
  
  The model parameter ``G'' for a Gibbs energy parameter.  All other
  must be associated with a particular model.  They are case
  insentitive.  The use of ``L'' for excess parameter should be
  discouraged and be converted to ``G''.

  Some MPID contain a reference to a constituent of the phase, for
  eample mobility data, and in the TDB file the species in included in
  the MPID separated by an ampersand ``\&''.  As XML uses this
  character for other things it is suggested this character is
  changed, for example to ``@''.

\item The hash character, ``\#'', frequently used in TDB files at the
  end of a function identifier, is redundant and not allowed.

  The hash character, ``\#'', followed by a digit, is also used in
  some software to indicate composition sets of a phase.

\item Any other use of special characters?

\end{enumerate}
  
\subsection{Detailed explanation of all XMLTDB elements and attributes}

For examples see also Appendix~\ref{sc:alfe}.

\subsubsection{Defaults element}\label{sc:defaults}

Some default values of particular attributes can be set here.
If an attribute has the default value it can be omitted for the
element.  If we have for example:
\begin{verbatim}
<Defaults low_T="298.15" high_T="6000" />
\end{verbatim}
this means that a low $T$ limit for a TPfun element such as:
\begin{verbatim}
<TPfun id="GHSERXX" low_T="298.15" >
<Trange high_T="6000" Expression=" 10000-10*T;" />
</TPfun>
\end{verbatim}
can be simplified to be:
\begin{verbatim}
<TPfun id="GHSERXX" Expression=" 10000-10*T;" />
\end{verbatim}
  
\subsubsection{Models element}

All models, except for the configurational entropy, whch are used in a
XMLTDB file must be taken from a general XMLTDB-models file which can
optionally be included in the XMLTDB file.  The XMLTDB-models file
have a short description of each model and specify the model parameter
identifiers (MPID) which should be provided.  Each model has a unique
ID which can be needed for adding this model to a phase, sometimes
with additional dat.  For example:

{\tiny
\begin{verbatim}
<Magnetic-model id="IHJBCC"  MPID1="TC" MPID2="BMAGN" anti-ferromagnetic_factor=" -1.00" 
   f_below_TC=" +1-0.905299383*TAO**(-1)-0.153008346*TAO**3-.00680037095*TAO**9-.00153008346*TAO**15;"
   f_above_TC=" -.0641731208*TAO**(-5)-.00203724193*TAO**(-15)-.000427820805*TAO**(-25);" bibref="82Her" > in G=f(TAO)*LN(BETA+1) where TAO=T/TC
</Magnetic-model>
<Magnetic-model id="IHJREST"  MPID1="TC" MPID2="BMAGN" anti-ferromagnetic_factor=" -3.00" 
   f_below_TC=" +1-0.860338755*TAO**(-1)-0.17449124*TAO**3-.00775516624*TAO**9-.0017449124*TAO**15;"
   f_above_TC=" -.0426902268*TAO**(-5)-.0013552453*TAO**(-15)-.000284601512*TAO**(-25);" bibref="82Her" > in G=f(TAO)*LN(BETA+1) where TAO=T/TC
</Magnetic-model>
<Magnetic-model id="IHJQX"  MPID1="CT" MPID2="NT" MPID3="BMAGN" anti-ferromagnetic_factor="  0.00" 
   f_below_TC=" +1-0.842849633*TAO**(-1)-0.174242226*TAO**3-.00774409892*TAO**9-.00174242226*TAO**15-.000646538871*TAO**21;"
   f_above_TC=" -.0261039233*TAO**(-7)-.000870130777*TAO**(-21)-.000184262988*TAO**(-35)-6.65916411E-05*TAO**(-49);" bibref="12Xiong" > in G=f(TAO)*LN(BETA+1) where TAO=T/CT or T/NT
</Magnetic-model>
<Einstein-model id="GLOWTEIN" MPID1="LNTH" bibref="01Qing" > 
   Gibbs energy due to the Einstein low T vibrational entropy model, G=1.5*R*THETA+3*R*T*LN(1-EXP(-THETA/T)).
</Einstein-model>
\end{verbatim}
}

Each model defined have a unique ID which is used together with the
phase element which have the model.  The MPIDs specified for the model
are used for the parameters describing how these parameters depend on
the phase constitution and possibly $T$ and $P$.

\subsubsection{Element element}

The element is from the periodic chart or can be fictitious.

An element has the attributes:
\begin{itemize}
\item a chemical symbol, one or two letters
\item the name of its reference phase
\item its mass in gram/mol
\item H298-H0 in J/mol
\item S298 in J/mol/K
\end{itemize}

{\small
\begin{verbatim}
<Element id="FE" refstate="BCC_A2" mass="55.847" H298="4489" S298="27.2" />
\end{verbatim}
}

Maybe add the English name of the element?  For example that Sb is
Antimony Sb?  Maybe remove H298-H0 and S298?

The vacancy is treated as an element and be used to occupy a site in a
sublattice.  When calculating an equilibrium the chemical potential
must be zero at equilibrium and there is no constraint on the number
of vacancies.

The electron, denoted ``/-'' can be listed among the elements but
cannot exist by itself.  If there is a need of a free electron it can
be associated with a vacancy.  At equilibrium each phases must be
electrically neutral.

All attributes are mandatory.

\subsubsection{Species element}\label{sc:spel}

A species is an element or a molecular like aggregate of two or more
elements with fixed stoichiometric ratios.  A species has a name which
is used when it is a constituent of a phase.  A species can have an
electric charge.  In the future some species used for phases with the
MQMQA~\cite{01Pel2} or UNIQUAC~\cite{20Li} models can have additional
data which is not yet elaborated.

\begin{verbatim}
<Species id="C1O" stoichiometry="C1O" />
<Species id="O-2" stoichiometry="O/-2" />
\end{verbatim}

One should avoid species names that could be interpreted as elements.
Species are used as constituent of the phases.  The stoichiometry must
contain element names followed by a real number as the stoichiometric
dactor.  Grouping using parenthesis is not allowed.  A charge is
specified using ``/-'' for a negative or ``/+'' for a positive ion.

Mandatory attributes are ``id'' and ``stoichiometry''.

\subsubsection{TPfun element}

Frequently some expressions are used in several model parameters and
it is convenient to define them separately as functions.  A TPfun has
an ID and a mathematical expression in $T$ and $P$ which can also
include other TPfuns.  Only very simple expressions are allowed but
can include references to other TPfun IDs.  For a simple example
see section~\ref{sc:defaults}

Thermodynamic data must frequently be extrapolated over a large range
of $T$ and $P$ and it can be necessary to breakpoint to change the
expression in different regions.  At such a breakpoint the values of
the expressions must have idenical values and well as identical first
and second derivatives (representing continuous entropy and heat
capacity).

To describe the expression with a limited range in $T$ or several
ranges the Trange element is used.

Mandatory attributes are ``id'', optional attribute is ``low\_T''
(which can be ignored if same as the default ``low\_T'') and
``Expression'' (which can be replaced by one or several ``Trange''
elements).  For the expression see section~\ref{sc:expression}

\subsubsection{Trange element}

This is part of a TPfun or Parameter element.  If it has a high\_T
limit different from the default it is specified together with the
expression, see section~\ref{sc:expression}.  Several ranges require
require several Trange elements.  The value of the function and its
first and second derivative wrt $T$ must be continuous when crossing a
range.  An example used for a TPfun element is:
{\small
\begin{verbatim}
<TPfun id="GHSERAL"  >
  <Trange High_T="700" Expression="-7976.15+137.093038*T-24.3671976*T*LN(T)-.001884662*T**2-8.77664E-07*T**3+74092*T**(-1);" />
  <Trange High_T="933.47" Expression="-11276.24+223.048446*T-38.5844296*T*LN(T)+.018531982*T**2-5.764227E-06*T**3+74092*T**(-1);" />
  <Trange High_T="2900" Expression="-11278.378+188.684153*T-31.748192*T*LN(T)-1.230524E+28*T**(-9);" />
</TPfun>
\end{verbatim}
}

Mandatory attribute is ``expression'' and optional ``high\_T''.

\subsubsection{Phase element}

This is the key element of the XMLTDB file.  All data are refered to a
specific phase.  A phase has a name, a configurational model depending
on its sublattices and constituents specified by extra elements.  All
model parameters belong to a phase. Phases can have additional models
specified.  For example:
\begin{verbatim}
  <Phase id="BCC_4SL" Configuration_model="CEF_3terms" state="S" >
    <Sublattices number_of="5"  Ratios="0.25  0.25  0.25  0.25  3" >
      <Constituent sublattice="1" list="AL FE" />
      <Constituent sublattice="2" list="AL FE" />
      <Constituent sublattice="3" list="AL FE" />
      <Constituent sublattice="4" list="AL FE" />
      <Constituent sublattice="5" list="VA" />
    </Sublattices>
    <Amend-phase model="IHJBCC" />
    <Amend-phase model="DISFS" info="BCC_4SL A2_BCC" sum_sublattices="4" />
    <Amend-phase model="BCCPERM" info="The parameters have BCC permutations" />
  </Phase>
\end{verbatim}
where the configurational\_model specify the Compound Energy Formalism
(CEF)~\cite{CEF} and that the Gibbs energy has 3 terms which is
associated with the disordered fraction model ``DISFS'' specified in
section~\ref{sc:disfs}.  The state is ``S'' for a solid phase which is
important if the Equi-Entropy Criterion (EEC) is applied at high $T$,
see section~\ref{sc:EEC}.

This phase has 5 sublattices where the first 4 represent an unsymetric
tetrahedron in the BCC lattice, all with AL and FE as constituents.
The last sublattices is for interstitials such as C or N but in this
case it is empty, i.e.  occupied only by VA representing the vacancy.
There are 3 times as many interstitial sites as substitutional in BCC.

One amend-phase element specifies a magnetic model with ``IHJBCC'' and
another the disordered fraction set model with ``DISFS'' which also
specify the ordered and disordered parts of the phase.  In OC this
is done differently but it may be useful to have an option to
extract just the disordered phase.

The ``BCCPERM'' model specifies that parameters which are identical
when constituents are permuted on the sublattices for ordering are
only listed once in the XMLTDB file.

Other configurational models used are I2SL for the partially ionic
2-sublattice model.  The CEF model include the ideal model without
sublattices and excess parameters and the single sublattice regular
model with Redlich-Kister-Muggianu (RKM) excess parameters as special
cases but ony may allow configurationsl models as ``ideal'' and
``RKM'' for these.

Models that are less frequent in TDB files, such as the modified
quasichemical model in the quadruple approximation (MQMQA) or UNIQUAC
for polymeric liquids, can be used for software which has implemented
them. However, they require also special species attribures, see
¨section~\ref{sc:spel}.

I do not think other binary excess models than Redlish-Kister (RK) is
interesting as one can always convert them to an RK model.  Ternary
excess models such as Kohler~\cite{60Koh} and Toop~\cite{65Toop} can
be interesting but one should consider Pelton's proposal that each
ternary can have a different extrapolation and implement that.  See
the section~\ref{sc:peltonexcess}.

Maybe there are other configurational models to think about for
example CVM?  It must be easy to add new models with new features
without changing the basic XMLTDB structure.

\subsubsection{Sublattices element}

This is part of the Phase element and specifies the number of
sublattices and the site ratio for each.  The sum of site ratios is
the formula unit of the phase.  Note that the number of atoms per
formula units can vary with composition.  Optionally more
crystallographic information can be provided but it will normally be
irrelevant for the thermodynamic calculations..

\subsubsection{Constituents element}

This is part of the Phase element and for each sublattice there is a
list of the constituents in the sublattice.  The order of the
sublattices must the the same as the order of sites in the ``ratio''
attribute of the sublattice element.  The order of the sublattices is
also important for thr parameter element, see~\ref{sc:parameter}.

\subsubsection{Amend-phase element}

This is part of the Phase element and specifies an additional model
for a phase.  It may have some extra model specifications, for example
in the ``DISFS'' model the ordered and disordered phases and the
number of sublattices to be summed are included.

\subsubsection{Parameter element}\label{sc:parameter}

This is the element which provides the actual data for the models of
the phases.  It has a complex identifier specifying the type of
property (MPID), the phase and sublattice constituents and a degree to
indicate some additional use.  Finally a bibliographic reference to
the paper where the parameter was published or who provided the
parameter.  The parameter must specify the phase because it may occur
independently of the phase as a scientist manually editing an XMLTDB
database may prefer to arrange all parameters for each binary or
ternary system together.

The complex identifier is the same which has been used in TDB files
and the phase constituents are in sublattice order, those mixing in
the same sublattice are separated by a comma, ``,'', and a colon,
``:'' separates constituents in different sublattices.

The parameter has a mathematical expression which can include TPfun
element names provided this is valid for the whole default $T$ range.
If it is valid in a limited $T$ range then one or more Trange elements
are needed to specify different expressions.  Examples:

{\small
\begin{verbatim}
<Parameter id="G(A2_BCC,AL:VA;0)"   Bibref="91Din" > 
  <Trange High_T="2900" Expression="+GALBCC;" />
</Parameter>

<Parameter id="G(BCC_4SL,AL:AL:AL:FE:VA;0)"   Expression="+GD03ALFE;" Bibref="08Sun" />
<Parameter id="TC(BCC_4SL,AL:AL:AL:FE:VA;0)"   Expression="-125;" Bibref="01Ohn" />
<Parameter id="BMAGN(BCC_4SL,AL:AL:AL:FE:VA;0)"   Expression="BMALFE;" Bibref="08Sun" />
\end{verbatim}
}

A parameter such as G(BCC\_4SL,AL:AL:AL:FE:VA;0) has 4 identical
permutations with the Fe in different sublattices.  As the ``BCCPERM''
model was indicated for the BCC\_4SL phase this parameter is listed
only once.  If the software does not have implemented these
permutations it has to generate them when reading the database.

The reason to use a compliex identifier is explained in
section~\ref{sc:manualedit} but in order to simplify the use of an XML
parser in the software one may use the element in
section~\ref{sc:parserpar}.


\subsubsection{Parameter2 element}\label{sc:parserpar}

This element has the same information as the parameter but is split in
parts in order to simplify the use of an XML parser.

\begin{verbatim}
<Parameter2 mpid="G" phase="FCC" bibref="90Gus">
  <Constituentarray degree="0">
    <Site refid="1">
      <Constituent id="W"/>
    </Site>
    <Site refid="2">
      <Constituent id="C"/>
      <Constituent id="VA"/>
   </Site>
  </Constituentarray>
  <Trange  expression=" 50000+10*T;" />
</Parameter>
\end{verbatim}

Software reading an XMLTDB file should be able to handle both forms of
the parameter element.

\subsubsection{Bibliography element}

There can be bibliographic references for each parameter and possibly
models for phases using bibitem elements.

\subsubsection{Bibitem element}

Inside the biblographic element each bibliographic reference should be
explicitly defined by this element, see Appendix~\ref{sc:alfe} for an
example.

\subsection{The mathematical expression}\label{sc:expression}

The mathematical expression used in expressions should be very simple.
No grouping of terms using parenthesis is allowed and division of
terms is not allowed.
\begin{itemize}
\item A simple term is a signed numeric value possibly multiplied with
  $T$ and $P$ raised to a power and possibly also multiplied with a
  TPfun symbol.
\item A complex term is a simple term multiplied with the natural
  logarithm, LN, or the exponential, EXP, of a simple term.
\item The argument of LN or EXP must be a simple term.
\item Negative powers of $T$ or $P$ must be enclosed in parenthesis.
\item As part of the Einstein low $T$ model a function GEIN is needed
  which can take a simple term as argument.  GEIN(X) is
  eq.~\ref{eq:egein} but the value of X should be the logarithm of the
  Einstein $\theta$, ln($\theta)$, because that is what is used in the
  composition dependent LNTH parameter.
\item In some software there are other functione allowed because they
  are used for kinetic models, for example ERF.
\end{itemize}

One reason to keep the expression simple is that some software needs
to calculate first and second derivatives of the Gibbs energy
expression with respect to $T$ and $P$ millions of times for each
equilibrium calculation.  Complex functions can be achived by
introducing several functions calling each other.  There is normally
no check of circular calls but such will normally lead to a crash of
the calculaion.

\subsection{Additional XML elements}

This consists for the moment only of model elements.  Each model has a
descriptive element with a reference.  Most models also have one or
more model parameter identifiers (MPID) which are used to describe how
the contribution to the Gibbs energy of the phase depend on $T, P$ and
its constitution.

\subsubsection{There are 3 magnetic models}

{\small
\begin{verbatim}
<Magnetic-model id="IHJBCC"  MPID1="TC" MPID2="BMAGN" anti-ferromagnetic_factor=" -1.00" 
  f_below_TC=" +1-0.905299383*TAO**(-1)-0.153008346*TAO**3-.00680037095*TAO**9-.00153008346*TAO**15;"
  f_above_TC=" -.0641731208*TAO**(-5)-.00203724193*TAO**(-15)-.000427820805*TAO**(-25);" bibref="82Her" > in G=f(TAO)*LN(BETA+1) where TAO=T/TC
</Magnetic-model>
<Magnetic-model id="IHJREST"  MPID1="TC" MPID2="BMAGN" anti-ferromagnetic_factor=" -3.00" 
  f_below_TC=" +1-0.860338755*TAO**(-1)-0.17449124*TAO**3-.00775516624*TAO**9-.0017449124*TAO**15;"
  f_above_TC=" -.0426902268*TAO**(-5)-.0013552453*TAO**(-15)-.000284601512*TAO**(-25);" bibref="82Her" > in G=f(TAO)*LN(BETA+1) where TAO=T/TC
</Magnetic-model>
<Magnetic-model id="IHJQX"  MPID1="CT" MPID2="NT" MPID3="BMAGN" anti-ferromagnetic_factor="  0.00" 
   f_below_TC=" +1-0.842849633*TAO**(-1)-0.174242226*TAO**3-.00774409892*TAO**9-.00174242226*TAO**15-.000646538871*TAO**21;"
   f_above_TC=" -.0261039233*TAO**(-7)-.000870130777*TAO**(-21)-.000184262988*TAO**(-35)-6.65916411E-05*TAO**(-49);" bibref="12Xiong" > in G=f(TAO)*LN(BETA+1) where TAO=T/CT or T/NT
</Magnetic-model>
\end{verbatim}
}

The functions is included only for documentation, they must be
implemeneted in the software to be used.  The MPID are slightly
different and different software may actually use different names.

There is also a question of the reference state for the magnetism.
Normally it is taken to be in the paramagentic state at high $T$.

{\em Maybe some extra term needed for the IHJQX model for the
  reference state?}

\subsubsection{The Einstein model}

This ie the Einstein model which depend on the Einstein $\theta$.  The
LNTH parameter should be the natural logariith of $\theta$.  This
adds a Gibbs energy contribution:
\begin{eqnarray}
G &=& 1.5R\theta + 3RT\ln( 1 - \exp(- \frac{\theta}{T})) \label{eq:egein}
\end{eqnarray}
where $\theta$ is the Einstein $T$.  But to have a better description
of the composition dependence the logarithm of $\theta$ is used in LNTH.

{\small
\begin{verbatim}
<Einstein-model id="GLOWTEIN" MPID1="LNTH" bibref="01Qing" > 
   Gibbs energy due to the Einstein low T vibrational entropy model, G=1.5*R*THETA+3*R*T*LN(1-EXP(-THETA/T)).
</Einstein-model>
\end{verbatim}
}

\subsubsection{The Liquid 2-state model}

The amorphous phase is an Einstein solid with $\theta$ in the LNTH
parameter and a G2 parameter determining the $T$ for the transition
from liquid to amorphous state.  This model removes the breakpoint in
the 1991 unary database when extrapolating the Gibbs energy of the
liquid to low $T$.  It does not describe the glas transition.

{\small
\begin{verbatim}
<Liquid-2state-model id="LIQ2STATE" MPID1="G2"  MPID2="LNTH" bibref="14Becker" >
   Unified model for the liquid and the amorphous state treated as an Einstein solid
</Liquid-2state-model>
\end{verbatim}
}

\subsubsection{A simple volume model}

It is also possible to make many parameters dependent on $P$ but the
result may be surprising.  This model was proposed by Lu et
al.~\cite{05Lu} for low $P<1$~GPa.  Note that each phase can have a
different volume model.
{\small
\begin{verbatim}
<Volume-model id="VOLOWP" MPID1="V0"  MPID2="VT" MPID3="VB" bibref="05Lu" >
   The volume of a phase is described as function of T, P and its constitution.
</Volume-model>
\end{verbatim}
}

The parameter V0 is a constant representing the volume at $T=298.15$
and $P=1$~bar.  VT is the thermal expansion at $P=1$~bar as function
of $T$.  VB represent a bulk modulus parameter wich can depend on $T$
and $P$ and should be fitted to high pressure data.

\subsubsection{The disordered fraction model}\label{sc:disfs}

This model simplifies modeling of phases with ordering and
order/disorder transitions.  It combines the Gibbs energy of the
ordered phase with another Gibbs energy without the ordering
sublattices (excluding any interstitial sublattice).  The interstitial
sublattice must be the last and, if present, the disordered phase must
also have an interstitial sublattice.

For a phase $\alpha$ with an ordered part, ($\alpha$,ord) and a
disordered one, ($\alpha$,dis) the Gibbs energy is calculated as
\begin{eqnarray}
G^{\alpha}_M = G^{\alpha,\rm dis}_M(x) -T~^{\rm cfg}S^{\alpha,\rm dis}_M + G^{\alpha,\rm ord}_M(y)
\end{eqnarray}
where the mole fractions $x$ are calculated from the ordered fractions
$y$.  This model is denoted in the configurational\_model attribute as
CEF\_2terms.  The configurational entropy of the disordered phase is
ignored.  It is preferred for modeling phases which never disorder
such as $\sigma, \mu$.

There is an alternative disordered model denoted as CEF\_3terms with
the expression
\begin{eqnarray}
G^{\alpha}_M = G^{\alpha,\rm dis}_M(x) + G^{\alpha,\rm ord}_M(y) - G^{\alpha,\rm ord}_M(y=x)
\end{eqnarray}
where the ordered part is calculated twice, the second time with the
mole fractions on all sublattices.  When the phase is disordered the
model parameters in $G^{\alpha,\rm ord}_M$ are eliminated.  This model
has been used for phases for order/disorder transformtions as the
parameters in the ordered phase does not affect the ordered phase but
it is now discouraged as the ordered parameters may create strange
phenomena.  It is denoted as CEF\_3terms.  In this model the disordered
configurational entropy is eliminated by default.

{\small
\begin{verbatim}
<Disordered-fraction-model id="DISFS" bibref="07Hal" >
   The disordered fractions are summed over the ordered sublattices indicated at the phase.  The Gibbs energy calculated 2 or 3 times as indicated by the CEF_appendix but the configurational entropy only once.
</Disordered-fraction-model>
\end{verbatim}
}

\subsubsection{The permutation models}

These model reduces the number of parameters in the database by
eliminate all permutations.  It must be implemented in the software.
BCC and FCC has different symmetry as the tetrahedra are different.

{\small
\begin{verbatim}
<FCC-permutations id="FCCPERM" bibref="09Sun" >
   Permutations of ordered FCC parameters with the same set of elements are listed only once.
</FCC-permutations>
\end{verbatim}
}

This model reduces the size of the database by avoiding storing all 24
permutations of a parameter such as G(FCC\_4SL,A:B:C:D:VA;0) of an
ordered FCC phase in the database.

{\small
\begin{verbatim}
<BCC-permutations id="BCCPERM" bibref="09Sun" >
   Permutations of ordered BCC parameters with the same set of elements are listed only once.
</BCC-permutations>
\end{verbatim}
}

The reduction of parameters is slightly less for BCC as the
tetrahedron is not symmetric.

\subsubsection{The EEC model}\label{sc:EEC}

The Equi-Entropy criteria~\cite{20Sun} means that a solid phase with
higher entropy than the liquid phase must not be allowed to become
stable.  At a given $T$ a solid phase can only have higher entropy
than a liquid (with different composition) if the extrapolation of the
Gibbs energy of the metastable solid is unphysical.  The aim of this
model is to remove the breakpoint in the Gibbs energy extrapolation of
solid phases above their melting $T$ and prevent the reappearence of
the solid at even higher $T$ due to an extrapolated heat capacity of
the metastable solid which has no physical meaning.

It must be implemented in the software but the database should
indicate if it is necessary.

{\small
\begin{verbatim}
<EEC id="EEC" bibref="20Sun" >
   Equi-Entropy Criterion means that solid phases with higher entropy that the liquid phase must not be stable.
</EEC>
\end{verbatim}
}


\subsubsection{Ternary excess models}\label{sc:peltonexcess}

Pelton~\cite{01Pel} has proposed a way to define the ternary excess
model separately for each ternary.  This could be implemented as an
amend-phase option.  Maybe someting like:
{\small
\begin{verbatim}
<Excess-model id="Kohler" bibref="60Kohler" >
   A ternary interaction has a Kohler excess model.  The three constituents are specified at the phase in any order.
</Excess-model>
<Excess-model id="Toop" bibref="65Toop" >
   A ternary interaction has a Toop excess model.  The three constituents are specified at the phase with the Toop constituent first.
</Excess-model>
\end{verbatim}
}

A phase with a Toop and Kohler model must for each ternary specify
{\small
\begin{verbatim}
<Amend-phase model="Toop" info="Fe Cr Ni" />
<Amend-phase model="Kohler" info="Mo Cr Ni" />
\end{verbatim}
}

For the Toop model the first of the 3 constituents is the Toop
constituent whereas for the Kohler model the order does not matter.

This is not implemented in any software as far as I know even if it
might work in GES5.


\setcounter{equation}{0}
\renewcommand{\theequation}{B\arabic{equation}} \setcounter{figure}{0}
\renewcommand{\thefigure}{B\arabic{figure}}
\section{Converting from TDB to XMLTDB format and vice versa}

An UPLOAD program~\cite{upload} has been developed and tested to
convert various TDB files to the XMLTDB format proposed here.  There
are still many features to be discussed to obtain the final format.

If it is found that the final XML format is too complex for manual
editing the TDB format may be kept for manual editing.  But that will
be a pain for the future.

\setcounter{equation}{0}
\renewcommand{\theequation}{C\arabic{equation}}
\setcounter{figure}{0}
\renewcommand{\thefigure}{C\arabic{figure}}
\section{Integration of 3rd generation models}

The new unary parameters tries to make the Calphad databases more
attractive to physicists and remove some of the less nice features of
the 2nd generation unary, such as the break points in the Gibbs
energies at the melting $T$ of the elements.

\subsection{The low $T$ heat capacity and entropy}

This concerns the Einstein model for low $T$ vibrational energy which
will also enforce the heat capacity is zero at $T=0$~K~\cite{01Chen}.
The Einstein heat capacity integrated to a Gibbs energy is shown in
eq.~\ref{eq:egein}.

The Einstein model was preferred as one anyway has to add a polynomial
in $T$ (without any powers of $T$ less or equal to 1 and no $\ln(T)$
term) in order to fit the deviation from the Dulong-Petit assumption
that the heat capaity at high $T$ should be $3R$.

The restriction that there should not be any linear term in $T$ is to
obey the 3rd law, i.e. that the entropy of a perfectly crystalline
phases must be zero at $T=0$~K.  But there is still a discussion if
phases with crystalline disorder, for example magnetite which is an
inverse spinel or the amorphous phase, should have $S=0$ at $T=0$~K.

There is an interesting use of the Einsten model for the lattice
stabilities (introduced by Larry Kaufman in 1970~\cite{70Kau} and
considered to be the first generation unary database) which represent
a Gibbs energy diffence between a stable, $\alpha$, and metastable,
$\beta$, state of a pure element, and which usually have a linear $T$
dependence, for example:
\begin{eqnarray}
  G^{\beta}_{\rm A} - G^{\alpha}_{\rm A} &=& \Delta H -T\Delta S
\end{eqnarray}
where the linear $T$ term can be converted to an Einstein $\theta$
and this have zero entropy at $T=0$~K.

Many of the lattice stabilities were kept in the 2nd generation
database and they are important to describe phase diagrams.  But they
violate the the 3rd law.  However, one can easily convert the $\Delta
S$ to an Einstein $\theta$ for the metastable phase using:
\begin{eqnarray}
  \theta^{\beta} &=& \theta^{\alpha}\exp(-\frac{\Delta S}{3R})
\end{eqnarray}
which will result in the same difference in Gibbs energy at finite $T$
and that metastable phases of the pure elements also has $S=0$ at
$T=0$~K.

\subsection{The magnetic model}

The second part of the new unary is the magnetic model with a new
expression and separate Curie and N{\'e}el $T$~\cite{12Xiong}.
Individual Bohr magneton numbers will most likely not implemented.  If
the reference state for the magnetic state is selected at the
paramagnetic state at high $T$ this may give some entropy at $T=0$~K.

Selecting the ferro- or antiferro magnetic state at $T=0$~K creates
other problems at high $T$.

\subsection{The low $T$ extrapolation of the liquid}

The liquid 2-state model~\cite{14Becker} use an Einstein solid for the
low $T$ amorphous state and a gradual transition to the stable high
$T$ liquid.  As already stated the entropy of the metastable amorphous
state may have nonzero entropy at $T=0$~K.

\subsection{The high $T$ extrapolation of the solid phase}

The extrapolation of the heat capacity of the solid phases of the pure
elements above their melting $T$ is modeled with a restricted
polynomial and this may extrapolate at high $T$ to a high heat
capacity and subsequently a very negative Gibbs energy of the solid.
This may be so negative that the solid would become more stable than
the liquid at even higher $T$.  This was avoided in the 1991 unary by
a breakpoint at the melting $T$ and forcing the solid heat capacity to
approach that of the liquid phase.  But this breakpoint leads to some
other problems, for example modeling compounds using the Kopp-Neumann
rule~\cite{Kopp-Neumann}, and should the breakpoint should be removed.

The Equi-Entropy Criteron (EEC)~\cite{20Sun} was introduced to handle
this extrapolation problem.  The EEC states that at a given $T$ a
solid phase cannot be in equilibrium, i.e. be stable, together with a
liquid if the solid has higher entropy than the liquid phase, even if
the solid and liquid has different compositions.

The fact that the solid phase has so high entropy is due to an
unphysical extrapolation of its heat capacity above its melting $T$.
The high entropy means the solid is ``mechanically unstable'' but to
find way to model this ``instability'' in multicomponent system has
turned out to be very complicated whereas the entropy criteria is
readily available.

\setcounter{equation}{0}
\renewcommand{\theequation}{D\arabic{equation}}
\setcounter{figure}{0}
\renewcommand{\thefigure}{D\arabic{figure}}
\section{Example of the tentative XMLTDB format}\label{sc:alfe}

A TDB file for the system Al-Fe has been used to generate this XML
file using the UPLOAD program.

\subsection{Example of the tentative XMLTDB format}

The lines are very long but the essential inoformation is still
readable.

{\tiny
\begin{verbatim}
<?xml version="1.0"?>
<?xml-model href="database.rng" schematypens="http://relaxng.org/ns/structure/1.0" type="application/xml"?>
<Database version="0.0.1">
  <metadata>
    <writer>xmltdbproject test
       TDBfile="C:\Users\bosun\Documents\GitHub\XMLTDB\software\examples\AlFe-4SLBF.TDB"
       Software="Thermo-Calc"
       Date="2023-05-15"
    </writer>
  </metadata>
  <!-- Statistics elements="4"  species="3"  tpfuns="26"  phases="9"  parameters="54"  bibrefs="6"  -->
  <Defaults low_T=" 298.15" high_T=" 6000.00" />
  <Models>
    <Magnetic-model id="IHJBCC"  MPID1="TC" MPID2="BMAGN" anti-ferromagnetic_factor=" -1.00" 
       f_below_TC=" +1-0.905299383*TAO**(-1)-0.153008346*TAO**3-.00680037095*TAO**9-.00153008346*TAO**15;"
       f_above_TC=" -.0641731208*TAO**(-5)-.00203724193*TAO**(-15)-.000427820805*TAO**(-25);" bibref="82Her" > in G=f(TAO)*LN(BETA+1) where TAO=T/TC
    </Magnetic-model>
    <Magnetic-model id="IHJREST"  MPID1="TC" MPID2="BMAGN" anti-ferromagnetic_factor=" -3.00" 
       f_below_TC=" +1-0.860338755*TAO**(-1)-0.17449124*TAO**3-.00775516624*TAO**9-.0017449124*TAO**15;"
       f_above_TC=" -.0426902268*TAO**(-5)-.0013552453*TAO**(-15)-.000284601512*TAO**(-25);" bibref="82Her" > in G=f(TAO)*LN(BETA+1) where TAO=T/TC
    </Magnetic-model>
    <Magnetic-model id="IHJQX"  MPID1="CT" MPID2="NT" MPID3="BMAGN" anti-ferromagnetic_factor="  0.00" 
       f_below_TC=" +1-0.842849633*TAO**(-1)-0.174242226*TAO**3-.00774409892*TAO**9-.00174242226*TAO**15-.000646538871*TAO**21;"
       f_above_TC=" -.0261039233*TAO**(-7)-.000870130777*TAO**(-21)-.000184262988*TAO**(-35)-6.65916411E-05*TAO**(-49);" bibref="12Xiong" > in G=f(TAO)*LN(BETA+1) where TAO=T/CT or T/NT
    </Magnetic-model>
    <Einstein-model id="GLOWTEIN" MPID1="LNTH" bibref="01Qing" > 
       Gibbs energy due to the Einstein low T vibrational entropy model, G=1.5*R*THETA+3*R*T*LN(1-EXP(-THETA/T)).
    </Einstein-model>
    <Liquid-2state-model id="LIQ2STATE" MPID1="G2"  MPID2="LNTH" bibref="14Becker" >
       Unified model for the liquid and the amorphous state treated as an Einstein solid
    </Liquid-2state-model>
    <Volume-model id="VOLOWP" MPID1="V0"  MPID2="VA" MPID3="VB" bibref="05Lu" >
       The volume of a phase is described as function of T, P and its constitution.
    </Volume-model>
    <Disordered-fraction-model id="DISFS" bibref="07Hal" >
       The disordered fractions are summed over the ordered sublattices indicated at the phase.  The Gibbs energy calculated 2 or 3 times as indicated by the CEF_appendix but the configurational entropy only once.
    </Disordered-fraction-model>
    <FCC-permutations id="FCCPERM" bibref="09Sun" >
       Permutations of ordered FCC parameters with the same set of elements are listed only once.
    </FCC-permutations>
    <BCC-permutations id="BCCPERM" bibref="09Sun" >
       Permutations of ordered BCC parameters with the same set of elements are listed only once.
    </BCC-permutations>
    <EEC id="EEC" bibref="20Sun" >
       Equi-Entropy Criterion means that solid phases with higher entropy that the liquid phase must not be stable.
    </EEC>
  </Models>
  <Element id="/-" refstate="ELECTRON_GAS" mass="  0.000000E+00" H298="  0.000000E+00" S298="  0.000000E+00" />
  <Element id="VA" refstate="VACUUM" mass="  0.000000E+00" H298="  0.000000E+00" S298="  0.000000E+00" />
  <Element id="AL" refstate="FCC_A1" mass="  2.698200E+01" H298="  4.577300E+03" S298="  2.832200E+01" />
  <Element id="FE" refstate="BCC_A2" mass="  5.584700E+01" H298="  4.489000E+03" S298="  2.728000E+01" />
  <Species id="VA" stoichiometry="VA" />
  <Species id="AL" stoichiometry="AL" />
  <Species id="FE" stoichiometry="FE" />
  <TPfun id="GHSERAL"  >
    <Trange High_T="700" Expression=" -7976.15+137.093038*T-24.3671976*T*LN(T)  -.001884662*T**2-8.77664E-07*T**3+74092*T**(-1);" />
    <Trange High_T="933.47" Expression="   -11276.24+223.048446*T-38.5844296*T*LN(T)+.018531982*T**2  -5.764227E-06*T**3+74092*T**(-1);" />
    <Trange High_T="2900" Expression="   -11278.378+188.684153*T-31.748192*T*LN(T)-1.230524E+28*T**(-9);" />
  </TPfun>
  <TPfun id="GALLIQ"  >
    <Trange High_T="933.47" Expression=" +11005.029-11.841867*T+7.934E-20*T**7  +GHSERAL;" />
    <Trange Expression="  +10482.382-11.253974*T+1.231E+28*T**(-9)+GHSERAL;" />
  </TPfun>
  <TPfun id="GALBCC"  Expression="+10083-4.813*T+GHSERAL;" />
  <TPfun id="GHSERFE"  >
    <Trange High_T="1811" Expression=" +1225.7+124.134*T-23.5143*T*LN(T)  -.00439752*T**2-5.8927E-08*T**3+77359*T**(-1);" />
    <Trange Expression="  -25383.581+299.31255*T-46*T*LN(T)+2.29603E+31*T**(-9);" />
  </TPfun>
  <TPfun id="GFELIQ"  >
    <Trange High_T="1811" Expression=" +12040.17-6.55843*T-3.6751551E-21*T**7  +GHSERFE;" />
    <Trange Expression="  -10839.7+291.302*T-46*T*LN(T);" />
  </TPfun>
  <TPfun id="GFEFCC"  >
    <Trange High_T="1811" Expression=" -1462.4+8.282*T-1.15*T*LN(T)+6.4E-04*T**2  +GHSERFE;" />
    <Trange Expression="  -27097.396+300.25256*T-46*T*LN(T)+2.78854E+31*T**(-9);" />
  </TPfun>
  <TPfun id="LFALFE0"  Expression="-104700+30.65*T;" />
  <TPfun id="LFALFE1"  Expression="+30000-7*T;" />
  <TPfun id="LFALFE2"  Expression="+32200-17*T;" />
  <TPfun id="UFALFE"  Expression="-4000+T;" />
  <TPfun id="GAL3FE"  Expression="+3*UFALFE+9000;" />
  <TPfun id="GAL2FE2"  Expression="+4*UFALFE;" />
  <TPfun id="GALFE3"  Expression="+3*UFALFE-3500;" />
  <TPfun id="SFALFE"  Expression="+UFALFE;" />
  <TPfun id="UBALFE1"  Expression="-4023-1.14*T;" />
  <TPfun id="UBALFE2"  Expression="-1973-2*T;" />
  <TPfun id="GD03ALFE"  Expression="+2*UBALFE1+UBALFE2+3900;" />
  <TPfun id="GB2ALFE"  Expression="+4*UBALFE1;" />
  <TPfun id="GB32ALFE"  Expression="+2*UBALFE1+2*UBALFE2;" />
  <TPfun id="GD03FEAL"  Expression="+2*UBALFE1+UBALFE2-70+0.5*T;" />
  <TPfun id="BMALFE"  Expression="-1.36;" />
  <TPfun id="BLALFE0"  Expression="-0.3;" />
  <TPfun id="BLALFE1"  Expression="-0.8;" />
  <TPfun id="BLALFE2"  Expression="0.2;" />
  <TPfun id="ZERO"  Expression="0.0;" />
  <TPfun id="UN_ASS"  >
    <Trange High_T="300" Expression=" 0.0 ;" />
  </TPfun>
  <Phase id="LIQUID" Configuration_model="CEF" state="L" >
    <Sublattices number_of="1"  Ratios="1" >
      <Constituent sublattice="1" list="AL FE" />
    </Sublattices>
  </Phase>
  <Phase id="A1_FCC" Configuration_model="CEF" state="S" >
    <Sublattices number_of="2"  Ratios="1  1" >
      <Constituent sublattice="1" list="AL FE" />
      <Constituent sublattice="2" list="VA" />
    </Sublattices>
    <Amend-phase model="IHJREST" />
  </Phase>
  <Phase id="A2_BCC" Configuration_model="CEF" state="S" >
    <Sublattices number_of="2"  Ratios="1  3" >
      <Constituent sublattice="1" list="AL FE" />
      <Constituent sublattice="2" list="VA" />
    </Sublattices>
    <Amend-phase model="IHJBCC" />
  </Phase>
  <Phase id="AL13FE4" Configuration_model="CEF" state="S" >
    <Sublattices number_of="3"  Ratios="0.6275  0.235  0.1375" >
      <Constituent sublattice="1" list="AL" />
      <Constituent sublattice="2" list="FE" />
      <Constituent sublattice="3" list="AL VA" />
    </Sublattices>
  </Phase>
  <Phase id="AL2FE" Configuration_model="CEF" state="S" >
    <Sublattices number_of="2"  Ratios="2  1" >
      <Constituent sublattice="1" list="AL" />
      <Constituent sublattice="2" list="FE" />
    </Sublattices>
  </Phase>
  <Phase id="AL5FE2" Configuration_model="CEF" state="S" >
    <Sublattices number_of="2"  Ratios="5  2" >
      <Constituent sublattice="1" list="AL" />
      <Constituent sublattice="2" list="FE" />
    </Sublattices>
  </Phase>
  <Phase id="AL8FE5_D82" Configuration_model="CEF" state="S" >
    <Sublattices number_of="2"  Ratios="8  5" >
      <Constituent sublattice="1" list="AL FE" />
      <Constituent sublattice="2" list="AL FE" />
    </Sublattices>
  </Phase>
  <Phase id="BCC_4SL" Configuration_model="CEF_3terms" state="S" >
    <Sublattices number_of="5"  Ratios="0.25  0.25  0.25  0.25  3" >
      <Constituent sublattice="1" list="AL FE" />
      <Constituent sublattice="2" list="AL FE" />
      <Constituent sublattice="3" list="AL FE" />
      <Constituent sublattice="4" list="AL FE" />
      <Constituent sublattice="5" list="VA" />
    </Sublattices>
    <Amend-phase model="IHJBCC" />
    <Amend-phase model="DISFS" info="BCC_4SL A2_BCC" sum_sublattices="4" />
    <Amend-phase model="BCCPERM" info="The parameters have BCC permutations" />
  </Phase>
  <Phase id="FCC_4SL" Configuration_model="CEF_3terms" state="S" >
    <Sublattices number_of="5"  Ratios="0.25  0.25  0.25  0.25  1" >
      <Constituent sublattice="1" list="AL FE" />
      <Constituent sublattice="2" list="AL FE" />
      <Constituent sublattice="3" list="AL FE" />
      <Constituent sublattice="4" list="AL FE" />
      <Constituent sublattice="5" list="VA" />
    </Sublattices>
    <Amend-phase model="IHJREST" />
    <Amend-phase model="DISFS" info="FCC_4SL A1_FCC" sum_sublattices="4" />
    <Amend-phase model="FCCPERM" info="The parameters have FCC permutations" />
  </Phase>
  <Unary-parameters >
    <Parameter id="G(LIQUID,AL;0)"   Expression="+GALLIQ;" Bibref="91Din" />
    <Parameter id="G(LIQUID,FE;0)"   Expression="+GFELIQ;" Bibref="91Din" />
    <Parameter id="G(A1_FCC,AL:VA;0)"   Expression="+GHSERAL;" Bibref="91Din" />
    <Parameter id="G(A1_FCC,FE:VA;0)"   Expression="+GFEFCC;" Bibref="91Din" />
    <Parameter id="TC(A1_FCC,FE:VA;0)"   Expression="-201;" Bibref="91Din" />
    <Parameter id="BMAGN(A1_FCC,FE:VA;0)"   Expression="-2.1;" Bibref="91Din" />
    <Parameter id="G(A2_BCC,AL:VA;0)"   Bibref="91Din" > 
      <Trange High_T="2900" Expression="+GALBCC;" />
    </Parameter>
    <Parameter id="G(A2_BCC,FE:VA;0)"   Expression="+GHSERFE;" Bibref="91Din" />
    <Parameter id="TC(A2_BCC,FE:VA;0)"   Expression="1043;" Bibref="91Din" />
    <Parameter id="BMAGN(A2_BCC,FE:VA;0)"   Expression="2.22;" Bibref="91Din" />
    <Parameter id="G(AL8FE5_D82,AL:AL;0)"   Expression="+13*GALBCC;" Bibref="08Sun" />
    <Parameter id="G(AL8FE5_D82,FE:FE;0)"   Expression="+13*GHSERFE+13000;" Bibref="08Sun" />
    <Parameter id="G(FCC_4SL,AL:AL:AL:AL:VA;0)"   Expression="+ZERO;" Bibref="08Con" />
    <Parameter id="G(FCC_4SL,FE:FE:FE:FE:VA;0)"   Expression="+ZERO;" Bibref="08Con" />
  </Unary-parameters>
  <Binary-parameters>
    <Parameter id="G(LIQUID,AL,FE;0)"   Expression="-88090+19.8*T;" Bibref="08Sun" />
    <Parameter id="G(LIQUID,AL,FE;1)"   Expression="-3800+3*T;" Bibref="08Sun" />
    <Parameter id="G(LIQUID,AL,FE;2)"   Expression="-2000;" Bibref="08Sun" />
    <Parameter id="G(A1_FCC,AL,FE:VA;0)"   Expression="+LFALFE0;" Bibref="08Sun" />
    <Parameter id="G(A1_FCC,AL,FE:VA;1)"   Expression="+LFALFE1;" Bibref="08Sun" />
    <Parameter id="G(A1_FCC,AL,FE:VA;2)"   Expression="+LFALFE2;" Bibref="08Sun" />
    <Parameter id="G(A2_BCC,AL,FE:VA;0)"   Expression="-122960+32*T;" Bibref="93Sei" />
    <Parameter id="G(A2_BCC,AL,FE:VA;1)"   Expression="2945.2;" Bibref="93Sei" />
    <Parameter id="TC(A2_BCC,AL,FE:VA;0)"   Expression="-438;" Bibref="01Ohn" />
    <Parameter id="TC(A2_BCC,AL,FE:VA;1)"   Expression="-1720;" Bibref="01Ohn" />
    <Parameter id="G(AL13FE4,AL:FE:AL;0)"   Expression="-30680+7.4*T+.765*GHSERAL  +.235*GHSERFE;" Bibref="08Sun" />
    <Parameter id="G(AL13FE4,AL:FE:VA;0)"   Expression="-28100+7.4*T+.6275*GHSERAL  +.235*GHSERFE;" Bibref="08Sun" />
    <Parameter id="G(AL2FE,AL:FE;0)"   Expression="-104000+23*T+2*GHSERAL+GHSERFE;" Bibref="08Sun" />
    <Parameter id="G(AL5FE2,AL:FE;0)"   Expression="-235600+54*T+5*GHSERAL  +2*GHSERFE;" Bibref="08Sun" />
    <Parameter id="G(AL8FE5_D82,FE:AL;0)"   Expression="+200000+36*T+5*GALBCC  +8*GHSERFE;" Bibref="08Sun" />
    <Parameter id="G(AL8FE5_D82,AL:FE;0)"   Expression="-394000+36*T+8*GALBCC  +5*GHSERFE;" Bibref="08Sun" />
    <Parameter id="G(AL8FE5_D82,AL:AL;0)"   Expression="-100000;" Bibref="08Sun" />
    <Parameter id="G(AL8FE5_D82,AL,FE:FE;0)"   Expression="-174000;" Bibref="08Sun" />
    <Parameter id="G(BCC_4SL,AL:AL:AL:FE:VA;0)"   Expression="+GD03ALFE;" Bibref="08Sun" />
    <Parameter id="TC(BCC_4SL,AL:AL:AL:FE:VA;0)"   Expression="-125;" Bibref="01Ohn" />
    <Parameter id="BMAGN(BCC_4SL,AL:AL:AL:FE:VA;0)"   Expression="BMALFE;" Bibref="08Sun" />
    <Parameter id="G(BCC_4SL,AL:AL:FE:FE:VA;0)"   Expression="+GB2ALFE;" Bibref="08Sun" />
    <Parameter id="TC(BCC_4SL,AL:AL:FE:FE:VA;0)"   Expression="-250;" Bibref="01Ohn" />
    <Parameter id="BMAGN(BCC_4SL,AL:AL:FE:FE:VA;0)"   Expression="2*BMALFE;" Bibref="08Sun" />
    <Parameter id="G(BCC_4SL,AL:FE:AL:FE:VA;0)"   Expression="+GB32ALFE;" Bibref="08Sun" />
    <Parameter id="TC(BCC_4SL,AL:FE:AL:FE:VA;0)"   Expression="-125;" Bibref="01Ohn" />
    <Parameter id="BMAGN(BCC_4SL,AL:FE:AL:FE:VA;0)"   Expression="BMALFE;" Bibref="08Sun" />
    <Parameter id="G(BCC_4SL,AL:FE:FE:FE:VA;0)"   Expression="+GD03FEAL;" Bibref="08Sun" />
    <Parameter id="TC(BCC_4SL,AL:FE:FE:FE:VA;0)"   Expression="-125;" Bibref="01Ohn" />
    <Parameter id="BMAGN(BCC_4SL,AL:FE:FE:FE:VA;0)"   Expression="BMALFE;" Bibref="08Sun" />
    <Parameter id="G(BCC_4SL,AL,FE:*:*:*:VA;1)"   Expression="-634+0.68*T;" Bibref="08Sun" />
    <Parameter id="G(BCC_4SL,AL,FE:*:*:*:VA;2)"   Expression="-190;" Bibref="08Sun" />
    <Parameter id="TC(BCC_4SL,AL,FE:*:*:*:VA;0)"   Expression="+125;" Bibref="01Ohn" />
    <Parameter id="BMAGN(BCC_4SL,AL,FE:*:*:*:VA;0)"   Expression="BLALFE0;" Bibref="08Sun" />
    <Parameter id="BMAGN(BCC_4SL,AL,FE:*:*:*:VA;1)"   Expression="BLALFE1;" Bibref="08Sun" />
    <Parameter id="BMAGN(BCC_4SL,AL,FE:*:*:*:VA;2)"   Expression="BLALFE2;" Bibref="08Sun" />
    <Parameter id="G(FCC_4SL,FE:AL:AL:AL:VA;0)"   Expression="+GAL3FE;" Bibref="08Con" />
    <Parameter id="G(FCC_4SL,FE:FE:AL:AL:VA;0)"   Expression="+GAL2FE2;" Bibref="08Con" />
    <Parameter id="G(FCC_4SL,FE:FE:FE:AL:VA;0)"   Expression="+GALFE3;" Bibref="08Con" />
    <Parameter id="G(FCC_4SL,AL,FE:AL,FE:*:*:VA;0)"   Expression="+SFALFE;" Bibref="08Con" />
  </Binary-parameters>
  <Bibliography>
    <Bibitem ID="91Din" Text="A T Dinsdale, Calphad 1991" />
    <Bibitem ID="93Sei" Text="M Seiersten, unpublished 1993" />
    <Bibitem ID="01Ohn" Text="I Ohnuma, unpublished 2001" />
    <Bibitem ID="08Con" Text="D Connetable et al, Calphad 2008; AL-C-Fe" />
    <Bibitem ID="08Sun" Text="B Sundman, to be published" />
    <Bibitem ID="08Dup" Text="N Dupin, vacancies in bcc" />
    <Bibitem ID="60Koh" Text="F. Kohler, Monatsh Chem, Vol 91 (1960) 738--740" />
    <Bibitem ID="65Toop" Text="G. W. Toop, Trans Metall Soc, AIME  Vol 233 (1965) 850--854" />
    <Bibitem ID="82Her" Text="S. Hertzman and B. Sundman, A Thermodynamic analysis of the Fe-Cr system, Calphad Vol 6 (1982) 67-80." />
    <Bibitem ID="12Xiong" Text="W. Xiong, Q. Chen, P. K. Korzhavyi and M. Selleby, An improved magnetic model for thermodynamic modeling, Calphad, Vol 39 (2012) 11-20." />
    <Bibitem ID="01Qing" Text="Q. Chen and B. Sundman, Modeling of thermodynamic properties for Bcc, Fcc, liquid, and amorphous iron, J. Phase Equilibria. Vol 22 (2001) 631-644." />
    <Bibitem ID="14Becker" Text="C. A. Becker, J. Agren, M. Baricco, Q. Chen, S. A. Decterov, U. R. Kattner, J. H. Perepezko, G. R. Pottlacher and M. Selleby, Thermodynamic modelling of liquids: CALPHAD approaches and contributions from statistical physics. Phys status solidi. B, Vol 251(1) (2014) 33-52." />
    <Bibitem ID="05Lu" Text="X-G. Lu, M. Selleby and B. Sundman, Implementation of a new model for pressure dependence of condensed phases in Thermo-Calc, Calphad Vol 29 (2005) 49-55." />
    <Bibitem ID="07Hal" Text="B. Hallstedt, N. Dupin, M. Hillert, L. Höglund, H. L. Lukas, J. C. Schuster and N. Solak, Thermodynamic models for crystalline phases, composition dependent models vor volume, bulk modulus and thermal expansion, Calphad Vol 31 (2007) 28-37" />
    <Bibitem ID="09Sun" Text="B. Sundman, I. Ohnuma, N. Dupin, U. R. Kattner, S. G. Fries, An assessment of the entire Al–Fe system including D03 ordering, Acta Mater. Vol 57 (2009) 2896-2908" />
    <Bibitem ID="20Sun" Text="B. Sundman, U. R. Kattner, M. Hillert, M. Selleby, J. Agren, S. Bigdeli, Q. Chen, A. Dinsdale, B. Hallstedt, A. Khvan, H. Mao and R. OtisA method for handling the estrapolation of solid crystalline phases to temperature above their melting point, Calphad Vol 68 (2020) 101737" />
  </Bibliography>
  <TDB-comments>
       0 $ Database file written 2008- 8-15
       0 $ From database: User data 2008. 8. 1
     127 $ THIS PHASE HAS A DISORDERED CONTRIBUTION FROM A2_BCC
     155 $ THIS PHASE HAS A DISORDERED CONTRIBUTION FROM A1_FCC
  </TDB-comments>
</Database>
\end{verbatim}
}

\subsection{The original TDB format}

{\small
\begin{verbatim}
$ Database file written 2008- 8-15
$ From database: User data 2008. 8. 1    
 ELEMENT /-   ELECTRON_GAS              0.0000E+00  0.0000E+00  0.0000E+00!
 ELEMENT VA   VACUUM                    0.0000E+00  0.0000E+00  0.0000E+00!
 ELEMENT AL   FCC_A1                    2.6982E+01  4.5773E+03  2.8322E+01!
 ELEMENT FE   BCC_A2                    5.5847E+01  4.4890E+03  2.7280E+01!
 
 FUNCTION GHSERAL   298.15 -7976.15+137.093038*T-24.3671976*T*LN(T)
     -.001884662*T**2-8.77664E-07*T**3+74092*T**(-1);  7.00000E+02  Y
      -11276.24+223.048446*T-38.5844296*T*LN(T)+.018531982*T**2
     -5.764227E-06*T**3+74092*T**(-1);  9.33470E+02  Y
      -11278.378+188.684153*T-31.748192*T*LN(T)-1.230524E+28*T**(-9);  
     2.90000E+03  N !
 FUNCTION GALLIQ    298.15 +11005.029-11.841867*T+7.934E-20*T**7
     +GHSERAL#;  9.33470E+02  Y
      +10482.382-11.253974*T+1.231E+28*T**(-9)+GHSERAL#; 6000  N !
 FUNCTION GALBCC    298.15 +10083-4.813*T+GHSERAL#;  6000   N      !
 FUNCTION GHSERFE   298.15 +1225.7+124.134*T-23.5143*T*LN(T)
     -.00439752*T**2-5.8927E-08*T**3+77359*T**(-1);  1.81100E+03  Y
      -25383.581+299.31255*T-46*T*LN(T)+2.29603E+31*T**(-9); 6000  N   !
 FUNCTION GFELIQ    298.15 +12040.17-6.55843*T-3.6751551E-21*T**7
     +GHSERFE#;  1.81100E+03  Y
      -10839.7+291.302*T-46*T*LN(T); 6000  N !
 FUNCTION GFEFCC    298.15 -1462.4+8.282*T-1.15*T*LN(T)+6.4E-04*T**2
     +GHSERFE#;  1.81100E+03  Y
      -27097.396+300.25256*T-46*T*LN(T)+2.78854E+31*T**(-9); 6000  N      !
 FUNCTION LFALFE0   298.15 -104700+30.65*T;  6000   N !
 FUNCTION LFALFE1   298.15 +30000-7*T;  6000   N !
 FUNCTION LFALFE2   298.15 +32200-17*T;  6000   N !
 FUNCTION UFALFE    298.15 -4000+T;  6000   N !
 FUNCTION GAL3FE    298.15 +3*UFALFE#+9000;  6000   N !
 FUNCTION GAL2FE2   298.15 +4*UFALFE#;  6000   N !
 FUNCTION GALFE3    298.15 +3*UFALFE#-3500;  6000   N !
 FUNCTION SFALFE    298.15 +UFALFE#;  6000   N !
 FUNCTION UBALFE1   298.15 -4023-1.14*T;  6000   N !
 FUNCTION UBALFE2   298.15 -1973-2*T;  6000   N !
 FUNCTION GD03ALFE  298.15 +2*UBALFE1#+UBALFE2#+3900;       6000   N !
 FUNCTION GB2ALFE   298.15 +4*UBALFE1#;  6000   N !
 FUNCTION GB32ALFE  298.15 +2*UBALFE1#+2*UBALFE2#;  6000   N !
 FUNCTION GD03FEAL  298.15 +2*UBALFE1#+UBALFE2#-70+0.5*T; 6000   N !
 FUNCTION BMALFE    298.15 -1.36; 6000 N !
 FUNCTION BLALFE0   298.15 -0.3; 6000 N !
 FUNCTION BLALFE1   298.15 -0.8; 6000 N !
 FUNCTION BLALFE2   298.15 0.2; 6000 N !
 FUNCTION ZERO      298.15 0.0; 6000.00  N !
 FUNCTION UN_ASS    298.15 0.0 ;  3.00000E+02  N !
 
 TYPE_DEFINITION % SEQ *!
 DEFINE_SYSTEM_DEFAULT ELEMENT 2 !
 DEFAULT_COMMAND DEF_SYS_ELEMENT VA /- !

 TYPE_DEFINITION X GES AMEND_PHASE_DESCRIPTION BCC_4SL DIS_PART A2_BCC,,,!
 TYPE_DEFINITION Y GES AMEND_PHASE_DESCRIPTION FCC_4SL DIS_PART A1_FCC,,,!

 PHASE LIQUID:L %  1  1.0  !
    CONSTITUENT LIQUID:L :AL,FE :  !

   PARAMETER G(LIQUID,AL;0) 298.15 +GALLIQ#;  6000   N 91Din !
   PARAMETER G(LIQUID,FE;0) 298.15 +GFELIQ#;  6000   N 91Din !
   PARAMETER G(LIQUID,AL,FE;0) 298.15 -88090+19.8*T;  6000     N 08Sun !
   PARAMETER G(LIQUID,AL,FE;1) 298.15 -3800+3*T;  6000   N   08Sun !
   PARAMETER G(LIQUID,AL,FE;2) 298.15 -2000;  6000   N 08Sun !


 TYPE_DEFINITION F GES A_P_D @ MAGNETIC  -3.0    2.80000E-01 !
 PHASE A1_FCC  %F  2 1   1 !
    CONSTITUENT A1_FCC  :AL,FE : VA :  !

   PARAMETER G(A1_FCC,AL:VA;0) 298.15 +GHSERAL#;  6000   N   91Din !
   PARAMETER G(A1_FCC,FE:VA;0) 298.15 +GFEFCC#;  6000   N   91Din !
   PARAMETER TC(A1_FCC,FE:VA;0) 298.15 -201;  6000   N 91Din !
   PARAMETER BMAGN(A1_FCC,FE:VA;0) 298.15 -2.1;  6000   N   91Din !
   PARAMETER G(A1_FCC,AL,FE:VA;0) 298.15 +LFALFE0#;  6000   N   08Sun !
   PARAMETER G(A1_FCC,AL,FE:VA;1) 298.15 +LFALFE1#;  6000   N   08Sun !
   PARAMETER G(A1_FCC,AL,FE:VA;2) 298.15 +LFALFE2#;  6000   N   08Sun !


 TYPE_DEFINITION B GES A_P_D @ MAGNETIC  -1.0    4.00000E-01 !
 PHASE A2_BCC  %B  2 1   3 !
    CONSTITUENT A2_BCC  :AL,FE : VA :  !

   PARAMETER G(A2_BCC,AL:VA;0) 298.15 +GALBCC#;  2.90000E+03  N 91Din !
   PARAMETER G(A2_BCC,FE:VA;0) 298.15 +GHSERFE#;  6000   N   91Din !
   PARAMETER TC(A2_BCC,FE:VA;0) 298.15 1043;  6000   N 91Din !
   PARAMETER BMAGN(A2_BCC,FE:VA;0) 298.15 2.22;  6000   N   91Din !
   PARAMETER G(A2_BCC,AL,FE:VA;0) 298.15 -122960+32*T;  6000     N 93Sei !
   PARAMETER G(A2_BCC,AL,FE:VA;1) 298.15 2945.2;  6000   N   93Sei !
   PARAMETER TC(A2_BCC,AL,FE:VA;0) 298.15 -438;  6000   N   01Ohn !
   PARAMETER TC(A2_BCC,AL,FE:VA;1) 298.15 -1720;  6000   N  01Ohn !



 PHASE AL13FE4  %  3 .6275   .235   .1375 !
    CONSTITUENT AL13FE4  :AL : FE : AL,VA :  !

   PARAMETER G(AL13FE4,AL:FE:AL;0) 298.15 -30680+7.4*T+.765*GHSERAL#
  +.235*GHSERFE#;  6000   N 08Sun !
   PARAMETER G(AL13FE4,AL:FE:VA;0) 298.15 -28100+7.4*T+.6275*GHSERAL#
  +.235*GHSERFE#;  6000   N 08Sun !


 PHASE AL2FE  %  2 2   1 !
    CONSTITUENT AL2FE  :AL : FE :  !

   PARAMETER G(AL2FE,AL:FE;0) 298.15 -104000+23*T+2*GHSERAL#+GHSERFE#;
    6000   N 08Sun !


 PHASE AL5FE2  %  2 5   2 !
    CONSTITUENT AL5FE2  :AL : FE :  !

   PARAMETER G(AL5FE2,AL:FE;0) 298.15 -235600+54*T+5*GHSERAL#
  +2*GHSERFE#;  6000   N 08Sun !


 PHASE AL8FE5_D82  %  2 8   5 !
    CONSTITUENT AL8FE5_D82  :AL,FE : AL,FE :  !

   PARAMETER G(AL8FE5_D82,AL:AL;0) 298.15 +13*GALBCC#;  6000     N 08Sun !
   PARAMETER G(AL8FE5_D82,FE:AL;0) 298.15 +200000+36*T+5*GALBCC#
  +8*GHSERFE#;  6000   N 08Sun !
   PARAMETER G(AL8FE5_D82,AL:FE;0) 298.15 -394000+36*T+8*GALBCC#
  +5*GHSERFE#;  6000   N 08Sun !
   PARAMETER G(AL8FE5_D82,FE:FE;0) 298.15 +13*GHSERFE#+13000; 6000   N 08Sun !
   PARAMETER G(AL8FE5_D82,AL:AL,FE;0) 298.15 -100000;    6000   N 08Sun !
   PARAMETER G(AL8FE5_D82,AL,FE:FE;0) 298.15 -174000;    6000   N 08Sun !


$ THIS PHASE HAS A DISORDERED CONTRIBUTION FROM A2_BCC                  
 PHASE BCC_4SL:B %BX  5 .25   .25   .25   .25   3 !
    CONSTITUENT BCC_4SL:B :AL,FE : AL,FE : AL,FE : AL,FE : VA :  !

   PARAMETER G(BCC_4SL,AL:AL:AL:FE:VA;0) 298.15 +GD03ALFE#;    6000   N 08Sun !
   PARAMETER TC(BCC_4SL,AL:AL:AL:FE:VA;0) 298.15 -125;  6000     N 01Ohn !
   PARAMETER BMAGN(BCC_4SL,AL:AL:AL:FE:VA;0) 298.15 BMALFE;  6000    N 08Sun !
   
   PARAMETER G(BCC_4SL,AL:AL:FE:FE:VA;0) 298.15 +GB2ALFE#;    6000   N 08Sun !
   PARAMETER TC(BCC_4SL,AL:AL:FE:FE:VA;0) 298.15 -250;    6000   N 01Ohn !
   PARAMETER BMAGN(BCC_4SL,AL:AL:FE:FE:VA;0) 298.15 2*BMALFE; 6000 N 08Sun !
   
   PARAMETER G(BCC_4SL,AL:FE:AL:FE:VA;0) 298.15 +GB32ALFE#;    6000   N 08Sun !
   PARAMETER TC(BCC_4SL,AL:FE:AL:FE:VA;0) 298.15 -125;  6000     N 01Ohn !
   PARAMETER BMAGN(BCC_4SL,AL:FE:AL:FE:VA;0) 298.15 BMALFE;  6000    N 08Sun !
   
   PARAMETER G(BCC_4SL,AL:FE:FE:FE:VA;0) 298.15 +GD03FEAL#;   6000   N 08Sun !
   PARAMETER TC(BCC_4SL,AL:FE:FE:FE:VA;0) 298.15 -125;  6000     N 01Ohn !
   PARAMETER BMAGN(BCC_4SL,AL:FE:FE:FE:VA;0) 298.15 BMALFE;  6000    N 08Sun !
   
   PARAMETER G(BCC_4SL,AL,FE:*:*:*:VA;1) 298.15 -634+0.68*T; 6000   N 08Sun !
   PARAMETER G(BCC_4SL,AL,FE:*:*:*:VA;2) 298.15 -190;    6000   N 08Sun !
   PARAMETER TC(BCC_4SL,AL,FE:*:*:*:VA;0) 298.15 +125;  6000     N 01Ohn !
   PARAMETER BMAGN(BCC_4SL,AL,FE:*:*:*:VA;0) 298.15 BLALFE0; 6000   N 08Sun !
   PARAMETER BMAGN(BCC_4SL,AL,FE:*:*:*:VA;1) 298.15 BLALFE1; 6000   N 08Sun !
   PARAMETER BMAGN(BCC_4SL,AL,FE:*:*:*:VA;2) 298.15 BLALFE2; 6000   N 08Sun !



$ THIS PHASE HAS A DISORDERED CONTRIBUTION FROM A1_FCC                  
 PHASE FCC_4SL:F %FY  5 .25   .25   .25   .25   1 !
    CONSTITUENT FCC_4SL:F :AL,FE : AL,FE : AL,FE : AL,FE : VA :  !

   PARAMETER G(FCC_4SL,AL:AL:AL:AL:VA;0) 298.15 +ZERO#;  6000     N 08Con !
   PARAMETER G(FCC_4SL,FE:AL:AL:AL:VA;0) 298.15 +GAL3FE#;    6000   N 08Con !
   PARAMETER G(FCC_4SL,FE:FE:AL:AL:VA;0) 298.15 +GAL2FE2#;   6000   N 08Con !
   PARAMETER G(FCC_4SL,FE:FE:FE:AL:VA;0) 298.15 +GALFE3#;    6000   N 08Con !
   PARAMETER G(FCC_4SL,FE:FE:FE:FE:VA;0) 298.15 +ZERO#;  6000     N 08Con !
   PARAMETER G(FCC_4SL,AL,FE:AL,FE:*:*:VA;0) 298.15 +SFALFE#;  6000   N 08Con !


ASSESSED_SYSTEM AL-FE(TDB -A2_B2 -A2_VA -BCC_VA -B2_BCC 
  ;G5 C_S:BCC_4/AL:AL:FE:FE:VA: C_S:BCC_4/AL:FE:FE:FE:VA:
  ;P3 TMM:300/3000 STP:0.99/1400/-1 STP:0.77/600/1 STP:0.45/500/1 )
!

 LIST_OF_REFERENCES
 NUMBER  SOURCE
 91Din 'A T Dinsdale, Calphad 1991'
 93Sei 'M Seiersten, unpublished 1993'
 01Ohn 'I Ohnuma, unpublished 2001'
 08Con 'D Connetable et al, Calphad 2008; AL-C-Fe'
 08Sun 'B Sundman, to be published'
 08Dup 'N Dupin, vacancies in bcc'
  ! 
\end{verbatim}
}
  
\end{appendices}

\end{document}


\subsubsection{All model parameter identifiers}

A recommended list of model parameter identifiers, the model they
are used in, if they can depend on $T$ and $P$ and their meaning.
These are the recommended for use in the XMLTDB files, each software
can use another their software.

\begin{tabular}{llll}
  MPID & Model name &  $T$ or $P$ dep & Represent\\\hline
  G    & -          &  $TP$ & Reference and excess Gibbs energy\\
  TC   & IHJBCC, IHJREST & $P$ & Curie or N{\'e}el $T$ \\
  BMAGN & IHJBCC, IHJREST, IHJQX & $P$ & Bohr magneton number \\
  CT   & IHJQX       & $P$ &Curie $T$ \\
  NT   & IHJQX       & $P$ & N{\'e}el $T$ \\
  LNTH & GLOWEIN     & none & Natural logarithm of Einstein $\theta$\\
  G2   & LIQ2STATE & $TP$  & Energy for amorphour state transition \\
  V0   & VOLOWP    & none & Volume at $T=298.15$~K and $P=1$~bar \\
  VA   & VOLOWP    & $T$  & Thermal expansition \\
  VB   & VOLOWP    & $TP$ & Related to the bulk modulus  \\\hline
  VS   & -         & ?    & Diffusion volume parameter \\
  MQ@  & -         & ?    & Mobility activation energy\\
  MF@  & -         & ?    & Mobility frequency factor\\
  MG@  & -         & ?    & Magnetic mobility factor\\\hline
\end{tabular}

}

